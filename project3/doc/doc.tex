# A Molecular Dynamics Code
\section{Molecular Dynamics: Input Reading and Distance Calculation}

\subsection{Input Handling}
The first part of the program focuses on reading the input data required for the molecular dynamics simulation. The input file \texttt{inp.txt} contains the number of atoms and their respective Cartesian coordinates and masses. With these, the following steps are implemented:

\begin{itemize}
    \item \textbf{Read Number of Atoms:}
    The function \texttt{read\_Natoms()} reads the total number of atoms specified in the first line of the input file.
    \item \textbf{Memory Allocation:}
    Using the function \texttt{malloc\_2d()}, memory is allocated dynamically for a two-dimensional array to store atomic coordinates (\(x, y, z\)) and another one for interatomic distances.
    \item \textbf{Input:}
    The \texttt{read\_molecule()} function reads the atomic coordinates and masses from the file and stores them in pre-allocated arrays.
\end{itemize}

\subsection{Distance Calculation}
The \texttt{compute\_distances()} function calculates the pairwise distances between all atoms using the Euclidean formula:

\begin{equation}
d_{ij} = \sqrt{(x_{i} - x_{j})^2 + (y_{i} - y_{j})^2 + (z_{i} - z_{j})^2}
\end{equation}

This is performed in a nested loop for all atom pairs, where the distance from an atom to itself is set to 0. The resulting distances are stored in a dynamically allocated two-dimensional array for further use in the simulation.

\subsection{Key Functions}
\begin{description}
    \item[\texttt{malloc\_2d():}] Allocates memory for a 2D array, ensuring efficient storage and access.
    \item[\texttt{read\_Natoms():}] Reads the total number of atoms from the input file.
    \item[\texttt{read\_molecule():}] Reads atomic coordinates and masses into allocated arrays.
    \item[\texttt{compute\_distances():}] Calculates and stores interatomic distances based on Cartesian coordinates.
\end{description}

\subsection{Validation and Testing}
The implementation also includes error handling, such as:
\begin{itemize}
    \item Checking for file-opening errors.
    \item Validating the successful allocation of memory.
    \item Handling unexpected input file formats or missing data.
\end{itemize}

The output is verified by printing the coordinates, masses, and calculated distances to the console for cross-checking.
